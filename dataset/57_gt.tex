\documentclass[a4paper,12pt]{article}
\usepackage[english]{babel}
\usepackage[utf8x]{inputenc}
\usepackage[T1]{fontenc}
\usepackage[flushmargin]{footmisc}
\usepackage{setspace}
\usepackage[comma]{natbib}
\usepackage{float}
\usepackage{amsmath}
\usepackage{amsfonts}
\usepackage{amssymb}
\usepackage{ae}
\usepackage{caption}
\usepackage[a4paper,top=3cm,bottom=2cm,left=3cm,right=3cm,marginparwidth=1.75cm]{geometry}
\usepackage{graphicx}
\usepackage[colorinlistoftodos]{todonotes}
\usepackage[colorlinks=true, allcolors=blue]{hyperref}

\begin{document} \doublespacing \pagestyle{plain}
\section{Proof of Theorem 3 }
\subsection{Average direct effect on the never-takers}
In the following, we show that $\theta_1^n= E[Y_1(1,0)-Y_1(0,0)|\tau=n]= E[Y_1-Q_{00}(Y_0)|D=1,M=0]$. From (), we obtain the first ingredient $E[Y_1(1,0)|\tau=n]=E[Y_1|D=1,M=0]$. Furthermore, from () we have $E[Q_{00}(Y_0)|D=1,M=0] = E[Y_1(0,0)|D=1,M(1)=0]$. Under Assumption 7 and 8,
\begin{equation}  \begin{array}{rl} E[Y_1(0,0)|D=1,M(1)=0]\stackrel{A7}{=}E[Y_1(0,0)|D=1,\tau=n]\\\stackrel{A8}{=}E[Y_1(0,0)|\tau=n].
\end{array} \end{equation}
\subsection{Quantile direct effect on the never-takers}
We prove that
\begin{align*}
\theta_1^n (q)= F_{Y_{1}(1,0)|\tau=n}^{-1}(q)- F_{Y_{1}(0,0)|\tau=n}^{-1}(q), \\
= F_{Y_1|D=1,M=0}^{-1}(q)-F_{Q_{00}(Y_{0})|D=1,M=0}^{-1}(q).
\end{align*}
 This requires showing that
\begin{align}
F_{Y_{1}(1,0)|\tau=n}(y) =F_{Y_1|D=1,M=0}(y) \mbox{ and} \\
F_{Y_{1}(0,0)|\tau=n}(y) = F_{Q_{00}(Y_{0})|D=1,M=0}(y). 
\end{align}
Under Assumptions 7 and 8,
\begin{equation} 
\begin{array}{rl}
F_{Y_t|D=1,M=0} (y) = E[1\{Y_t\leq y\}|D=1,M=0] \\ \stackrel{A7,A8}{=}E[1\{Y_t(1,0)\leq y\}|\tau=n] \\ = F_{Y_{t}(1,0)|\tau=n} (y), \end{array}
\end{equation}
which proves (). From (), we have
\begin{equation*}
F_{Q_{00}(Y_{0})|D=1,M=0}(y) = F_{Y_{1}(0,0)|D=1,M(1)=0} (y) = E[1\{Y_1(0,0) \leq y\}|D=1,M(1)=0].
\end{equation*}
\end{document}
