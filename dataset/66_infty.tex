\documentclass[a4paper,12pt]{article}
\usepackage{latexsym}
\usepackage{amsmath}
\usepackage{amssymb}
\usepackage{graphicx}
\usepackage{wrapfig}
\pagestyle{plain}
\usepackage{fancybox}
\usepackage{bm}

\begin{document}

At each simulation step $t$ and for each ?, the path length reflected by

the ?-th scatter {\it lt}(?) $= |(x_{t},\ y_{t})-(0,0)| + |(x_{t}^{*}\ ,\ y_{t}^{*}\ ) -(x_{t},\ y_{t})|$ and its deriva-

tive with respect to time $\displaystyle \frac{d}{dt}$ {\it l}(?) are computed. The corresponding transmis-

sion delay time s(?) , the constant phase offset ?(?) , and the Doppler frequency

{\it f}(?t) caused by the ?-th scatterer following the rules s(?) $=$ {\it l}(?)/{\it c}0, ?(?) $=$

((-{\it l}(?){\it f}carrier/{\it c}0) $\mathrm{m}\mathrm{o}\mathrm{d}\ 1$).2p, and $f_{D_{\mathrm{t}}}^{(\ovalbox{\tt\small REJECT})} =-\displaystyle \frac{d}{dt}$ {\it l}(?){\it f}carrier/{\it c}0, respectively, as well

as the received signal amplitude {\it at}(?) computed using the free-space propagation

model {\it at}(?) $=$ {\it c}0/(4p{\it f}carrier{\it l}(?)) (cf.) are recorded for each scatterer ?. (Here,

$c_{0}$ refers to the speed of light in vacuum.) In asetting without line of sight,

using linearisation of the phase offset with respect to the Doppler frequency,

the time-variant channel impulse response evaluated at time $ t+Д$ for each sim-

ulation step $t$ and small t resulting from the multipath transmission simulated

using the above parameters can be approximated by

255

$h (\cdot,\ t+Д)= \displaystyle \frac{1}{\sqrt{\sum_{\ovalbox{\tt\small REJECT}--0}^{255}(a_{t}^{(\ovalbox{\tt\small REJECT})})^{2}}}\sum_{\ovalbox{\tt\small REJECT}=0}$ {\it at}(?) exp({\it i}?{\it t}(?)$+${\it i}2p{\it fD}(?t)t)ds(?) $(\cdot)$ .

For any signal $\{S_{Д}\}_{0\leqД<T}$ being transmitted in the block beginning at time

step $t$ through the simulated channel, this consideration leads to a received

signal $\{R_{Д}\}_{0\leqД<T}$ in the form of
\begin{center}
$R$t $= (h\ (\cdot,\ t\ +\ Д)\ *\ S.)(Д)$
\end{center}
255
\begin{center}
$= \displaystyle \frac{1}{\sqrt{\sum_{\ovalbox{\tt\small REJECT}--0}^{255}(a_{t}^{(\ovalbox{\tt\small REJECT})})^{2}}}\sum_{\ovalbox{\tt\small REJECT}=0}$ {\it at}(?) exp({\it i}?{\it t}(?)$+${\it i}2p{\it fD}(?t)t)(dst(?) $(\cdot)*S.$) $(Д)$ .   (1)
\end{center}
This parametrisation is used in and delivers a realistic approximation of real-

world scenarios for numbers of summands greater than 100 . In order to allow

continuous time delays to be applied to discrete time signals, the impulse func-

tion d st(?) $(\cdot)$ in $()$ is convolved with a windowed sinc $(\cdot)$ function scaled with

a given bandwidth. Overall, the channel transmission including pulse shaping

with bandwidth restricted to half the sample rate

and additive noise is approximated by replacing the d s(?) $(\cdot)$ in $()$ by

sin(p($\cdot$/2))/ $(A(\cdot/2))1_{[-8,8]}$ and adding independent and identically distributed

Gaussian white noise $\sim \mathcal{N}$ ($0$, s2) to the transmitted signal with powers2 re-

sulting in a signal-to-noise ratio of $12\mathrm{d}\mathrm{B}.$

1 Channel Estimation

The time-variant channel transfer functions $\mathcal{F}h(\cdot,\ t+Д)$ for $t = 0$, . . . , $4095T$

and $0 \leqД < T$ simulated in Section are approximated by a time series of

block wise time-invariant transfer functions $\{\mathcal{F}h^{t}\}_{t=0,\ldots,4095}$ based on which

the estimation and prediction of the channel transmission are conducted. For

each transmission block beginning at time step $t$, in order to estimate the

corresponding channel transfer function $\mathcal{F}h^{t}$, a complex-valued (white noise)

test signal $\{\overline{S}_{Д}^{t}\}_{Д=0,\ldots,N-1}$ whose Fourier transform has constant amplitude

and random phases $\sim \mathcal{U}$(-p, p) is generated and then transmitted through the

channel simulated in Section resulting in a received signal $\{R_{Д}^{t}\}_{Д=0,\ldots,N-1}.$

1
\end{document}
