\documentclass[a4paper,12pt]{article}
\usepackage{latexsym}
\usepackage{amsmath}
\usepackage{amssymb}
\usepackage{graphicx}
\usepackage{wrapfig}
\pagestyle{plain}
\usepackage{fancybox}
\usepackage{bm}

\begin{document}

0.0.1 Summary of first-order equations

To summarize, the first-order equations for the warp factor $f$ and the scalars $\sigma, \phi^{0}, \phi^{3}$ are

found to be
$$
f'=2(G_{0}S_{0}+G_{3}S_{3})
$$
$$
\sigma'=2\eta\sqrt{N_{0}^{2}+N_{3}^{2}}
$$
$$
\cos\phi^{3}(\phi^{0})'=-(G_{0}M_{0}+G_{3}M_{3})
$$
\begin{center}
$(\phi^{3})'=i(G_{3}M_{0}-G_{0}M_{3})$   (1)
\end{center}
Furthermore, for consistency these were required to satisfy the algebraic constraint
\begin{center}
$e^{-2f}=4(G_{0}S_{0}+G_{3}S_{3})^{2}-4(S_{0}^{2}+S_{3}^{2})$   (2)
\end{center}
The various functions featured in these equations were defined in and

0.1 Numeric solutions

In order to get acceptable numerical solutions from these equations, we must choose ap-

propriate initial conditions. It is easy to check that the following initial conditions ensure

smoothness of all three scalars, as well as the vanishing of $e^{2f}$ at the origin,

$\phi_{0}^{3}=\sin^{-1} [_{\overline{8}}$tanlh $\phi_{0}^{0}$ ($-3+\sqrt{9+16}$tanh2 $\phi_{0}^{0}$)]
\begin{center}
$\displaystyle \sigma_{0}=\frac{1}{4}\log\ [\frac{\cosh\phi_{0}^{0}(5+\sqrt{9+16\tanh^{2}\phi_{0}^{0}})}{6\sqrt{8+\coth^{2}\phi_{0}^{0}(-3+\sqrt{9+16\tanh^{2}\phi_{0}^{0}})}}]$   (3)
\end{center}
We have defined for notational convenience $\phi_{0}^{\alpha} \equiv \phi^{\alpha}(0)$ and $\sigma_{0} \equiv \sigma(0)$ . For these initial

conditions to be real, we must ensure that
\begin{center}
$|f(\phi_{0}^{0})| \leq 1 f(\phi_{0}^{0})\equiv_{\overline{8}}$tanlh $\phi_{0}^{0}$ ($-3+\sqrt{9+16}$tanh2 $\phi_{0}^{0}$)   (4)
\end{center}
Noting that

$\displaystyle \lim f(\displaystyle \phi_{0}^{0})=-\frac{1}{4}$

$\phi_{0}^{0}\rightarrow-\infty$

$\displaystyle \lim f(\displaystyle \phi_{0}^{0})=\frac{1}{4}$ (5)

$\phi_{0}^{0}\rightarrow+\infty$

and also that $f(\phi_{0}^{0})$ is monotonically increasing, i.e.
\begin{center}
$\displaystyle \frac{df}{d\phi_{0}^{0}}>0\ \forall\phi_{0}^{0}\in \mathbb{R}$   (6)
\end{center}
1
\end{document}
