\documentclass[a4paper,12pt]{article}
\usepackage{latexsym}
\usepackage{amsmath}
\usepackage{amssymb}
\usepackage{graphicx}
\usepackage{wrapfig}
\pagestyle{plain}
\usepackage{fancybox}
\usepackage{bm}

\begin{document}

under treatment randomization as in successfully conducted experiments. This

allows identifying the ATE: $\Delta_{1} = E[Y_{1}|D\ =\ 1] -E[Y_{1}|D\ =\ 0]$. Furthermore, we

assume the mediator to be weakly monotonic in the treatment.

Assumption 8: Weak monotonicity of the mediator in the treatment.

$\mathrm{P}\mathrm{r}(M(1)\geq M(0))=1.$

Assumption 8 is standard in the instrumental variable literature on local aver-

age treatment effects when denoting by $D$ the instrument and by $M$ the endoge-

nous regressor, see and . It rules out the existence of defiers. As discussed in

the Appendix , the total ATE $\Delta_{1} = E[Y_{1}|D\ =\ 1] - E[Y_{1}|D\ =\ 0]$ and QTE

$\Delta_{1}(q) =F_{Y_{1}|D=1}^{-1}(q)-F_{Y_{1}|D=0}^{-1}(q)$ for the entire population are identified under As-

sumption 7. Furthermore, Assumptions 7 and 8 yield the strata proportions, de-

noted by $p_{Д} =$ Pr(t), as functions of the conditional mediator probabilities given

the treatment, which we denote by $p_{(m|d)} = \mathrm{P}\mathrm{r} (M\ =\ m|D\ =\ d)$ for {\it d}, $m \in \{0$, 1$\}$

(see Appendix ):
\begin{center}
$pa=p_{1|0},\ pc=p_{1|1}-p_{1|0}=p_{0|0}-p_{0|1},\ pn=p_{0|1}$.   (1)
\end{center}
Furthermore, Assumptions 2, 7, and 8 imply that (see Appendix )
\begin{center}
$\displaystyle \Delta_{0,c}=E[Y_{0}(1,1)-Y_{0}(0,0)|c]\ =\frac{E[Y_{0}|D=1]-E[Y_{0}|D=0]}{p_{1|1}-p_{1|0}}=0$.   (2)
\end{center}
Therefore, a rejection of the testable implication $E[Y_{0}|D\ =\ 1]-E[Y_{0}|D\ =\ 0] = 0$

in the data would point to a violation of these assumptions. Assumptions 7 and

8 permit identifying additional parameters, namely the total, direct, and indirect

effects on compliers, and the direct effects on never- and always-takers, as shown in

Theorems 3 to 5. This follows from the fact that defiers are ruled out and that the

proportions and potential outcome distributions of the various principal strata are

not selective w.r. $\mathrm{t}$. the treatment.

Theorem 3: Under Assumptions $1\ovalbox{\tt\small REJECT} 3$, 7-8,
\end{document}
