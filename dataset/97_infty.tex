\documentclass[a4paper,12pt]{article}
\usepackage{latexsym}
\usepackage{amsmath}
\usepackage{amssymb}
\usepackage{graphicx}
\usepackage{wrapfig}
\pagestyle{plain}
\usepackage{fancybox}
\usepackage{bm}

\begin{document}

questioning the presence of $(i,j)$ in each structure. By construction, the

distribution of $X_{i,j}$ is given by the base pairing probability $\mathbb{P}(X_{i,j}(s)=1)=p_{i,j}$ .

Any base pair, $(i,j)$ , has an {\it entropy}, defined by the information entropy of $X_{i,j}$ ,

i.e.
$$
H(X_{i,j})=-p_{i,j}\log_{2}p_{i,j}-(1-p_{i,j})\log_{2}(1-p_{i,j})\ ,
$$
where the units of $H$ are in bits. The entropy $H (X_{i,j}\ )$ measures the uncertainty

of the base pair $(i,\ j)$ in $\Omega$ . When abase pair $(i,\ j)$ is certain to either exist

or not, its entropy $H (X_{i,j}\ )$ is $0$. However, in case $p_{i,j}$ is closer to 1/2, $H(X_{i,j})$

becomes larger. The r.v. $X_{i,j}$ partitions the space $\Omega$ into two disjoint sub-

spaces $\Omega_{0}$ and $\Omega_{1}$, where $\Omega_{k} = \{s\ \in\ \Omega\ :\ X_{i,j}(s)\ =\ k\} (k\ =\ 0,1)$ , and the

induced distributions are given by

$p_{0}(s)=\displaystyle \frac{p(s)}{1-p_{i,j}}$ for{\it s} $\in\Omega_{0}, p_{1}(s)=\displaystyle \frac{p(s)}{p_{i,j}}$ for{\it s} $\in\Omega_{1}.$

Intuitively, $H(X_{i,j})$ quantifies the average bits of information we would expect

to gain about the ensemble when querying a base pair $(i,j)$ This motivates

us to consider the {\it maximum entropy base pairs}, the base pair $(i_{0},j_{0})$ having

maximum entropy among all base pairs in $\Omega$, i.e.
$$
(i_{0},j_{0})={\rm argmax} H(X_{i,j})(i,j).
$$
As we shall prove in Section, $X_{i_{0},j_{0}}$ produces maximally balanced splits.

0.1. {\it The ensemble tree}

Equipped with the notion of ensemble and bit query (i.e. the respective max-

imum entropy base pairs), we proceed by describing our strategy to identify the

target structure as specified in Problem. The first step consists in having a

closer look at the space of ensemble reductions. Each split obtained by parti-

tioning the ensemble $\Omega$ using r.v. $X_{i,j}$, can in turn be bipartitioned itself via

any of its maximum entropy base pairs. This recursive splitting induces the

{\it ensemble tree}, $T(\Omega)$ , whose vertices are sub-samples and in which its {\it k}-th layer

represents a partition of the original ensemble into $2^{k}$ blocks. $T(\Omega)$ , isarooted

binary tree, in which

$\epsilon_{i,j} ((x_{1},\ .\ .\ .\ x_{j}),\ (y_{1},\ .\ .\ .\ y_{m})) = (y_{1},\ .\ .\ .\ y_{i-1},\ x_{1},\ .\ .\ .\ x_{j},\ y_{i+1},\ .\ .\ .\ y_{m})$
\begin{center}
?$i,j (x_{1},\ .\ .\ .\ x_{n}) = (\ (x_{i},\ .\ .\ .\ x_{j}),\ (x_{1},\ .\ .\ .\ x_{i-1},\ x_{j+1},\ .\ .\ .\ x_{n}))$ .
\end{center}
To quantify to what extent modularity can discriminate base pairs, we perform

computational experiments on random sequences via splittings. For each se-

quence, we consider its MFE structure $s$ computed via ViennaRNA (?) . Given

two positions $i$ and $j$, we cut the entire sequence $\mathrm{x}$ into two fragments, $\mathrm{x}_{i,j}$ and

the remainder $\overline{\mathrm{x}}_{i,j}$, i.e., ?{\it i}, $j(\mathrm{X}) = (\mathrm{x}_{i,j}\ ,\ \overline{\mathrm{x}}_{i,j})$ . Subsequently, the two fragments

$\mathrm{x}_{i,j}$ and $\overline{\mathrm{x}}_{i,j}$ refold into their MFE structures $s_{i,j}$ and $\overline{s}_{i,j}$, respectively, which

are combined into a structure $\epsilon_{i,j} (s_{i,j}\ ,\ \overline{s}_{i,j})$ . If bases{\it i} and{\it j} are paired in $s$

1
\end{document}
