\documentclass[a4paper,12pt]{article}
\usepackage[english]{babel}
\usepackage[utf8x]{inputenc}
\usepackage[T1]{fontenc}
\usepackage[flushmargin]{footmisc}
\usepackage{setspace}
\usepackage[comma]{natbib}
\usepackage{float}
\usepackage{amsmath}
\usepackage{amsfonts}
\usepackage{amssymb}
\usepackage{ae}
\usepackage{caption}
\usepackage[a4paper,top=3cm,bottom=2cm,left=3cm,right=3cm,marginparwidth=1.75cm]{geometry}
\usepackage{graphicx}
\usepackage[colorinlistoftodos]{todonotes}
\usepackage[colorlinks=true, allcolors=blue]{hyperref}

\begin{document} \doublespacing \pagestyle{plain}
under treatment randomization as in successfully conducted experiments. This allows identifying the ATE: $\Delta_1 = E[Y_1|D=1] -E[Y_1|D=0]$.
Furthermore, we assume the mediator to be weakly monotonic in the treatment.\vspace{5 pt}\\
\textbf{Assumption 8:} Weak monotonicity of the mediator in the treatment.\\
$\Pr(M(1) \geq M(0)) =1.$\vspace{5 pt}\\
Assumption 8 is standard in the instrumental variable literature on local average treatment effects when denoting by $D$ the instrument and by $M$ the endogenous regressor, see  and . It rules out the existence of defiers.
As discussed in the Appendix , the total ATE $\Delta_1= E[Y_1 |D=1]- E[Y_1 |D=0]$ and QTE $\Delta_1(q) = F_{Y_{1} |D=1}^{-1}(q)- F_{Y_{1} |D=0}^{-1}(q)$ for the entire population are identified under Assumption 7. Furthermore, Assumptions 7 and 8 yield the strata proportions, denoted by $p_{\tau}= \Pr(\tau)$, as functions of the conditional mediator probabilities given the treatment, which we denote by $p_{(m|d)}=\Pr(M=m|D=d)$ for $d, m \in \{0,1\}$ (see Appendix ):
\begin{equation} 
p_a =p_{1|0}, p_c = p_{1|1}-p_{1|0} = p_{0|0}-p_{0|1}, p_n = p_{0|1}.
\end{equation}
Furthermore, Assumptions 2, 7, and 8 imply that (see Appendix )
\begin{equation} 
\Delta_{0,c} =E[Y_0(1,1) - Y_0(0,0)|c] = \frac{E[Y_0|D=1] -E[Y_0|D=0] }{p_{1|1} - p_{1|0}} =  0.
\end{equation}
Therefore, a rejection of the testable implication $E[Y_0|D=1] -E[Y_0|D=0]=0$ in the data would point to a violation of these assumptions.
Assumptions 7 and 8 permit identifying additional parameters, namely the total, direct, and indirect effects on compliers, and the direct effects on never- and always-takers, as shown in Theorems 3 to 5. This follows from the fact that defiers are ruled out and that the proportions and potential outcome distributions of the various principal strata are not selective w.r.t.\ the treatment.\\
\noindent \textbf{Theorem 3:} Under Assumptions 1–3, 7-8,
\end{document}
