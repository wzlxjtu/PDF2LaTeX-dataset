\documentclass[a4paper,12pt]{article}
\usepackage{latexsym}
\usepackage{amsmath}
\usepackage{amssymb}
\usepackage{graphicx}
\usepackage{wrapfig}
\pagestyle{plain}
\usepackage{fancybox}
\usepackage{bm}

\begin{document}

solar thermal on-site do not have $CO_{2}$ associated. Embed-

ded emissions were not included. For additional details on the

model and the references of the input data used in the model,

refer to .

I. GRID TARIFFS DESCRIPTION

The Norwegian electricity consumption’s recent trend is a

consumption where peak demand increases relatively more

than annual demand. This trend must be met by new incentives

to shave peak load in order to avoid costly distribution grid

investments. Grid tariffs are one effective way to solve this

issue. In this paper we suggest three new grid tariffs and

compare the results with the current grid tariff. The first

analyzed grid tariff is energy based and is the current tariff

in Norway. It consists of an annual fixed price and a grid

energy cost per kWh consumed. As this rate is flat, it does

not incentivize flexible resources nor consumption patterns

which results in lower peak demand. The annual cost can be

calculated using $($??$)$ .
\begin{center}
$C^{tot}=137+0$, 0225 $\displaystyle \sum_{t} y_{t}^{imp}$   (1)
\end{center}
The second grid tariff is a time-of-use based tariff which

penalizes import when there is typically scarcity in the grid.

The tariff has a basic cost, which is double during peak load

hours (7-10am and 6-9pm) and reduced to half during low load

hours (11pm-5am). The effect of increasing electric vehicle

and demand response penetration on the peak hours is ignored.

The total costs are given by $($??$)$ .

$C^{tot}=\displaystyle \sum_{t}(0$,0123 $y_{t}^{imp}-\ low+0$,0246 $y_{t}^{imp}$

$+0$, 0492 $y_{t}^{imp}$ {\it peak}) (2)

The third tariff was originally described in , and is called

capacity subscription. It contains a fixed annual cost (e/year),

a capacity cost $(\mathrm{e}/\mathrm{k}\mathrm{W})$ , an energy cost $(\mathrm{e}/\mathrm{k}\mathrm{W}\mathrm{h})$ and an excess

demand charge $(\mathrm{e}/\mathrm{k}\mathrm{W}\mathrm{h})$ . The energy cost is significantly

higher when the imports are above the subscription. The main

advantage of this tariff is that it incentivizes peak shaving

and creates a market for capacity where consumers pay for

the resource which in fact is scarce in the distribution grid:

capacity. Disadvantages are complexity and the uncertainty

in consumer behaviour. In addition, the optimal subscribed

capacity is unknown in advance. Finding its value is further

discussed in . In this paper, the subscribed capacity is a

variable in the optimization. In reality, the consumer would

have to choose it and it would most likely not be the optimal

value. The costs are calculated with $($??$)$ .

$C^{tot} = 108\displaystyle \cdot c^{sub}+\sum_{t}$ ($0,005\cdot y_{t}^{imp-\ }${\it below} $+0, 1\cdot y_{t}^{imp-\ }${\it above})

(3)

The fourth tariff is a dynamic tariff where grid scarcity is

taken into account. As an extra incentive to reduce impacts

on the grid, apenalization $C^{sc}$ is given for consumption in

hours with grid scarcity. Scarcity d{\it tc} in the system is defined

as the 5\% of hours in the region (NO1) when the load is the

highest. The percentage chosen is arbitrary and could be tuned

or changed into a threshold by the regulator. The total costs

are given by $($??$)$ . In addition, as an added incentive to help the

grid, a bonus for exporting in those hours is added, at the same

cost as the scarcity tariff. In $($??$)$ , d{\it tsc} is abinary parameter

defining for each hour if there is scarcity in the grid.

$C^{tot}=\displaystyle \sum_{t}($($0$,0225 (1-- d{\it tsc}) $+$d{\it tsc} $0,1$) $\cdot y_{t}^{imp_{-}tot}$

$-0, 1$. d{\it tsc} $y_{t}^{exp- tot}$) (4)

II. RESULTS

In Norway, the legislation regarding prosumers is changing,

moving from a situation where exports are limited to $100\mathrm{k}\mathrm{W}$

to a situation of unrestrained export. For this reason, both

cases are investigated to explore the consequences on the

design of ZENs of the different grid tariffs in these cases.

The investment in the energy system can be seen in Table

and in Table , respectively for the case without and with

limitation on exports. The results are presented in the format

Prod Plant/ Student Housing/ Normal Offices/ Passive Offices.

The investments stay similar, no new technology is introduced

or replaced. However, small variations in the amount of each

technology appear, in particular heat storage. The difference

between the energy system with and without export limit

is greater, namely due to storages. A large battery pack is

necessary in order to store the PV production while it waits to

be exported, i.e. to accommodate the bottleneck. In addition,

large investments in heat storages and electric boilers are done.

The subscribed capacity resulting of the optimization is of

134, $5\mathrm{k}\mathrm{W}$ for the case with no export limit, and of $124\mathrm{k}\mathrm{W}$

in the case with export limits. Fig. presents the total cost of

the neighborhood’s energy system (investment and operation)

and the total revenue for the DSO, both over the lifetime and

discounted to the start of the study. There are small variations

in the cost in all cases. Subscribed capacity and dynamic

pricing cause an increase in the total cost for the ZEN between

3 and 5\% compared with the energy case. On the other hand,

the time of use scheme allows for a cost reduction of around

12\% in the case without export limit and 5\% with export

limit. The DSO revenue from the ZEN are higher when using

the other pricing schemes than with the energy scheme when

there is no export limit. When there is export limits, the DSO

revenue stays the same because the battery allows to self-

consume more and ”anticipates” the higher price periods and

buys electricity when the price is lower. The revenue in the

case of export limits are about half of the revenue of the case of

no export limit except in the case of subscribed capacity where

the subscription tariff allows to maintain the revenue. The cost

increase in the ZEN is of the same order of magnitude as the

increase in revenue for the DSO except for $\mathrm{T}\mathrm{o}\mathrm{U}$ where the cost

of the ZEN decreases while the revenue for the DSO increases.

$\mathrm{T}\mathrm{o}\mathrm{U}$ has a beneficial effect from both points of view in this

aspect. The duration curves Fig. , in the case of no export

limit, are not affected much by the tariff scheme in place.
\end{document}
