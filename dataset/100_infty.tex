\documentclass[a4paper,12pt]{article}
\usepackage{latexsym}
\usepackage{amsmath}
\usepackage{amssymb}
\usepackage{graphicx}
\usepackage{wrapfig}
\pagestyle{plain}
\usepackage{fancybox}
\usepackage{bm}

\begin{document}

over the conditioning feature $X_{1}$, i.e.
$$
H(X_{2}|X_{1})=\sum_{\mathrm{i}_{1}}\mathbb{P}(X_{1}=x_{\mathrm{i}_{1}}^{(1)})H(X_{2}|X_{1}=x_{\mathrm{i}_{1}}^{(1)})\ .
$$
Proposition 1 (Chain rule,).
\begin{center}
$H(X_{1},\ X_{2})=H(X_{1})+H(X_{2}|X_{1})$ .   (1)
\end{center}
Proposition 2. {\it Let R} $($?, $X_{1}, X_{2})$ {\it denote the} entropy reduction {\it of}?fi{\it rst by the}

{\it feature} $X_{1}$ {\it and then by the feature} $X_{2}$, {\it and R} $($?, $(X_{1},\ X_{2}))$ {\it denote the} entropy

reduction {\it of} ? {\it by a pair of features} $(X_{1},\ X_{2})$ . {\it Then}
\begin{center}
$R ($?, $X_{1}, X_{2})=R(\ovalbox{\tt\small REJECT},\ (X_{1},\ X2))$ .   (2)
\end{center}
The maximum entropy of an arbitrary feature is achieved when all its outcomes

occur with equal probability, and this maximum value is proportional to the

logarithm of the number of possible outcomes to the base 2. Thus Proposition

implies that the more possible outcomes a feature has, the higher entropy reduc-

tion it could possibly lead to. Meanwhile, afeature with an arbitrary number of

outcomes can be viewed as a combination of {\it binary features}, the ones with two

possible outcomes. Even though the entropy of the combination of two features

is greater than each of them, Proposition shows that partitioning the space sub-

sequently by two features has the same entropy reduction as partitioning by

their combination. Therefore, instead of considering features with outcomes as

many as possible, we focus on binary features.

1. Query repeats

Here we assess the improvement of the error rates by repeating the same

query twice. Let $Y$ (or $N$) denote the event of the queried base pair existing

(or not) in the target structure. Let $y$ (or {\it n}) denote the event of the experi-

ment confirming (or rejecting) the base pair. Let {\it nn} denote the event of two

independent experiments both rejecting the base pair. Similarly, we have {\it yy}

and {\it yn}. Utilizing the same sequences and structures as described in Fig., we

estimate the conditional probabilities $\mathbb{P}(n|N)\approx 0.993$ and $\mathbb{P}(n|Y) \approx 0.055$. The

{\it prior probability} $\mathbb{P} (Y\ )$ can be computed via the expected number $l_{1}$ of confirmed

queried base pairs on the path, divided by the number of queries in each sam-

ple. Fig. displays the distribution of {\it l}1 having mean around5. Thus we adopt

$\mathbb{P} (Y\ ) =\mathbb{P}(N)=0.5$. By Bayes’theorem, we calculate the {\it posterior}
$$
\mathbb{P}(N|nn)=\frac{\mathbb{P}(nn|N)\mathbb{P}(N)}{\mathbb{P}(nn)}=\frac{\mathbb{P}(n|N)^{2}\mathbb{P}(N)}{\mathbb{P}(n|N)^{2}\mathbb{P}(N)+\mathbb{P}(n|Y)^{2}\mathbb{P}(Y)},
$$
where $\mathbb{P}(nn) = \mathbb{P}(nn|N)\mathbb{P}(N)+\mathbb{P}(nn|Y)\mathbb{P}(Y\ )$ . we have $\mathbb{P}(nn|N) = \mathbb{P}(n|N)^{2}$

and $\mathbb{P} (nn|Y) =\mathbb{P}(n|Y)^{2}$. Similarly, we compute $\mathbb{P}(Y|nn)$ , $\mathbb{P}(Y|yy)$ and $\mathbb{P}(Y|yn)$

1
\end{document}
