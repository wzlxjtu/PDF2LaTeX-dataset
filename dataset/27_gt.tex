\documentclass[12pt]{article}
\setlength{\topmargin}{-.3in}
\setlength{\oddsidemargin}{0in}
\setlength{\textheight}{8.2in}
\setlength{\textwidth}{6.5in}
\setlength{\footnotesep}{\baselinestretch\baselineskip}
\newlength{\abstractwidth}
\setlength{\abstractwidth}{\textwidth}
\addtolength{\abstractwidth}{-6pc}
\usepackage{amsmath}
\usepackage{amsfonts}
\usepackage{amssymb}
\usepackage{latexsym}
\usepackage{epsf}
\usepackage{color}
\usepackage{graphicx}
\usepackage{tikz}
\usepackage{dsfont}
\usepackage{subfigure}
\usepackage{hyperref}
\pagestyle{plain}
\begin{document}


These will turn out to be the same, so we just work with the former. Thus we have that 
\begin{eqnarray}
\bar \psi_A = \psi_A^\dagger \gamma_7
\end{eqnarray}
If we choose $\gamma_7$ such that 
\begin{eqnarray}
(\gamma_7)^\dagger = - \gamma_7
\end{eqnarray}
we can express the Hermitian conjugates of our gamma matrices as
\begin{eqnarray}
\gamma_\mu^\dagger = \eta\, G^{-1} \gamma_\mu G
\end{eqnarray}
Importantly, with $G=G_1$ in , we have 
\begin{eqnarray}
\eta = -1
\end{eqnarray}
This will be important in Appendix  when the consistency of the symplectic Majorana condition is analyzed. 
For now, we just recall that the symplectic Majorana condition must take the form 
\begin{eqnarray}
\bar \psi_A = \epsilon^{AB} \psi_B^T\, {\cal C}
\end{eqnarray}
where 
\begin{eqnarray}
{\cal C}^2 = 1 \hspace{0.7 in} {\cal C}^T = {\cal C} \hspace{0.7in} \gamma_\mu^T = - {\cal C}^{-1} \gamma_\mu {\cal C}
\end{eqnarray}
We now want to reduce from $d=7$ to $d=6$. In particular, we reduce on the time-like direction $x_7$. This entails finding a Euclidean induced metric on the six-dimensional surface . From the point of view of the Clifford algebra, we must remove the matrix $\gamma_7$ to get a six-dimensional Clifford algebra. However, the properties of the matrix $\gamma^7$ remain the same. In fact, we may choose
\begin{eqnarray}
\gamma_7 = \gamma_0\gamma_1\gamma_2\gamma_3\gamma_4\gamma_5
\end{eqnarray}
which satisfies all of the properties ,.
\section{Free differential algebra}
\setcounter{equation}{0}
In this Appendix, we will construct the free differential algebra (FDA) of a supergravity theory with $\mathbb{H}_6$ background in order to motivate the form of the supersymmetry variations given in .
The first step of constructing the FDA is to write down the Maurer-Cartan equations (MCEs), which may be thought of as the geometrization of the (anti-)commutation relations of the superalgebra. In short, instead of defining the algebra via the (anti-)commutators of its generators, the MCEs encode the algebraic structure in integrability conditions. In 
\end{document}
