\documentclass[a4paper,12pt]{article}
\usepackage{amsmath}
\usepackage{amsthm}
\usepackage{setspace}
\usepackage{graphicx}
\usepackage{authblk}
\usepackage{amsfonts}
\usepackage{natbib}
\bibliographystyle{plainnat}
\usepackage{xr}
\usepackage{hyperref}
\usepackage[toc,page]{appendix}
\usepackage{enumitem}

\begin{document}
with zero corresponding to a NA value. Next we want to identify the bivariate outliers by iterating the filter over all possible pairs of variables. Consider a pair of variables $\bf{X}^{(jk)} = \{ (X_{ij},X_{ik}) \}, i = 1, \ldots, n$. The initial location and dispersion estimators are, respectively, the coordinate-wise median and the $2 \times 2$ sub-matrix $S^{(jk)}$ of the estimate $S$ computed by the generalized S-estimator on non-filtered data. Note that, this ensure the positive definiteness property for $S$ and each $d \times d$ sub-matrix corresponding to a subset of $d$ variables. For bivariate points with no flagged components by the univariate filter we compute the squared Mahalanobis distance $\Delta^{(jk)}_i$ and hence apply the bivariate filter, for all $1 < j < k < p$. At the end we want to identify the cells $(i,j)$ which have to be flagged as cell-wise outliers. The procedure used for this purpose is described in \citet{Zamar2017} and reported here. Let
\begin{equation*}
J = \{ (i,j,k) : \Delta_i^{(jk)} \mbox{ is flagged as bivariate outlier} \}
\end{equation*}
be the set of triplets which identifies the pairs of cells flagged by the bivariate filter in rows $i = 1, \ldots, n$. For each cell $(i,j)$ in the data, we count the number of flagged pairs in the $i$-th row in which the considered cell is involved:
\begin{equation*}
m_{ij} = \#\{ k : (i,j,k) \in J\}.
\end{equation*}
In absence of contamination, $m_{ij}$ follows approximately a binomial distribution $Bin(\sum_{k \not = j}\bf{U}_{jk},\delta)$ where $\delta$ represents the overall proportion of cell-wise outliers undetected by the univariate filter. Hence, we flag the cell $(i,j)$ if $m_{ij} > c_{ij}$, where $c_{ij}$ is the $0.99$-quantile of $Bin(\sum_{k \not = j}\bf{U}_{jk},0.1)$.
Finally, we perform the $p$-variate filter as described in subsection  to the full data matrix. Detected observations (rows) are directly flagged as $p$-variate (case-wise) outliers. We denote the procedure based on univariate, bivariate and $p$-variate filters, HS-UBPF.
\subsection{A  sequencing filtering procedure}
Suppose we would like to apply a sequence of $k$ filters with different dimension $1 \le d_1 \le d_2 \le \ldots \le d_k \le p$. For each $d_i$, $i = 1, \ldots, k$, the filter updates the data matrix adding NA values to the $d_i$-tuples identified as $d_i$-variate outliers. In this way, each filter applies only those $d_i$-tuples that have not been flagged as outliers by the filters with lower dimension.
Initial values for each procedures rather than $d_1$ would be obtained by applying the GSE to the actual filtered values.
This procedure aims to be a valid alternative to that used in the presented HS-UBPF filter to perform a sequence of filters with different dimensions. However, this is a preliminary idea, indeed it has not been implemented yet.
\end{document}
