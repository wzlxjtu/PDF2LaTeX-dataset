\documentclass[a4paper,12pt]{article}
\usepackage{latexsym}
\usepackage{amsmath}
\usepackage{amssymb}
\usepackage{graphicx}
\usepackage{wrapfig}
\pagestyle{plain}
\usepackage{fancybox}
\usepackage{bm}
\begin{document}
where $\omega_k$ are
\begin{gather*}
 \omega_0=1 , \qquad \omega_1=-1 , \qquad \omega_2=p^\frac12 ,\qquad \omega_3=-p^{-\frac12} .
\end{gather*}
The functions $p_k(h)$ are
\begin{alignat*}{3}
  p_0(h)\equiv \prod_n \theta(p^\frac12 h_n) , \qquad  p_1(h)\equiv \prod_n \theta\big({-}p^\frac12 h_n\big) , \\
  p_2(h)\equiv p\prod_n h_n^{-\frac12}\theta(h_n) ,\qquad  p_3(h)\equiv p\prod_n h_n^{\frac12}\theta\big({-}h_n^{-1}\big) ,
\end{alignat*}
and $\mathcal{E}_k$ is
\begin{gather*}
 \mathcal{E}_k(\xi;z)\equiv\frac{\theta\big(q^{-\frac12} \xi \omega_k^{-1} z\big)\theta\big(q^{-\frac12} \xi \omega_k z^{-1}\big)}{\theta\big(q^{-\frac12}\omega_k^{-1} z\big)\theta\big(q^{-\frac12} \omega_k z^{-1}\big)} .
\end{gather*}
The van Diejen operator and the operator () are the same up to a constant function (independent of~$z$). It's clear that $V(h;z)$ coincides with the corresponding term in \eqref{diuer} if we make the identifications
\begin{gather*}
 h_{1,2,3,4}=t^{-1}A^{\pm1}C^{\pm1} , \qquad h_{5,6,7,8}=t^{-1}B^{\pm1}D^{\pm1} .
\end{gather*}
Since $V_b(h;z)$ is elliptic in $z$ with periods~$1$ and $p$ and it is easy to check that $W^{{\mathfrak J}_B}_{{\mathfrak J}_D, (1,0; AB^{-1})}(z)$ is also elliptic with the same period, it is enough to show that the two functions have the same poles and residues to prove that they can differ only by a function independent of~$z$. In the fundamental parallelogram $V_b$ has poles at (we assume with no loss of generality that $|p|<|q|\ll |t|<1$ and the rest of the variables are on unit circle)
\begin{gather*}
 z=\pm q^{-\frac12}p , \qquad z= \pm q^{\frac12}, \qquad z=\pm p^\frac12 q^{\pm\frac12}.
\end{gather*}
In addition to such poles the operator \eqref{diuer} seems to have poles at $z=\pm t^{-2}p , \pm t^2 , 	\pm p^\frac12 t^{\pm2}$ and $z=\pm1,\pm p^\frac12$, but computation of the residue at these poles yields zero. The computation of the residue at the poles is straightforward, the result is ($h$ is either $1$ or $-1$)
\begin{gather*}
\operatorname{Res}_{z\to h q^\frac12} W^{{\mathfrak J}_B}_{{\mathfrak J}_D, (1,0; AB^{-1})}(z)=
-h(p;p)^{-2}\frac{\theta_p\big(h p^\frac{1}{2}t^{\pm1}A C^{\pm1}\big)\theta_p\big(h p^\frac{1}{2}t^{\pm1}B^{-1}D^{\pm1}\big)}
{2q^{-\frac12}\theta_p\big(q^{-1}\big)},\\
\operatorname{Res}_{z\to h q^{-\frac12}} W^{{\mathfrak J}_B}_{{\mathfrak J}_D, (1,0; AB^{-1})}(z)=
h(p;p)^{-2}\frac{\theta_p\big(hp^\frac{1}{2}t^{\pm1}A C^{\pm1}\big)\theta_p\big(h p^\frac{1}{2}t^{\pm1}B^{-1}D^{\pm1}\big)}
{2q^{\frac12}\theta_p\big(q^{-1}\big)},\\
\operatorname{Res}_{z\to h p^\frac12 q^{\frac12}} W^{{\mathfrak J}_B}_{{\mathfrak J}_D, (1,0; AB^{-1})}(z)=
-h (p;p)^{-2}\frac{A^{-2}B^{2}\theta_p\big(h t^{\pm1}A C^{\pm1}\big)\theta_p\big(h t^{\pm1}B^{-1}D^{\pm1}\big)}
{2p^{-\frac32}q^{-\frac12}\theta_p\big(q^{-1}\big)},\\
\operatorname{Res}_{z\to h p^\frac12 q^{-\frac12}} W^{{\mathfrak J}_B}_{{\mathfrak J}_D, (1,0; AB^{-1})}(z)=
h (p;p)^{-2}\frac{A^{-2}B^{2}\theta_p\big(h t^{\pm1}A C^{\pm1}\big)\theta_p\big(h t^{\pm1}B^{-1}D^{\pm1}\big)}
{2p^{-\frac32}q^{\frac12}\theta_p\big(q^{-1}\big)}.
\end{gather*}
\end{document} 
