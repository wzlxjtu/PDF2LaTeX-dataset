\documentclass[12pt]{article}
\setlength{\topmargin}{-.3in}
\setlength{\oddsidemargin}{0in}
\setlength{\textheight}{8.2in}
\setlength{\textwidth}{6.5in}
\setlength{\footnotesep}{\baselinestretch\baselineskip}
\newlength{\abstractwidth}
\setlength{\abstractwidth}{\textwidth}
\addtolength{\abstractwidth}{-6pc}
\usepackage{amsmath}
\usepackage{amsfonts}
\usepackage{amssymb}
\usepackage{latexsym}
\usepackage{epsf}
\usepackage{color}
\usepackage{graphicx}
\usepackage{tikz}
\usepackage{dsfont}
\usepackage{subfigure}
\usepackage{hyperref}
\pagestyle{plain}
\begin{document}

allows us to conclude that this is always the case for real initial conditions $\phi^0_0$. Thus we have a one parameter family of real smooth solutions, labeled by the IR parameter $\phi^0_0$.
With this in mind, we may choose any value of $\phi^0_0$ and solve the BPS equations in numerically. In Figure , we plot the solutions obtained for the following choices of initial condition: $\phi^0_0=\{0.25,\,0.5,\,1,\,1.5,\,2 \}$. In order to get smooth solutions for $u>0$, we must take $\eta = -1$. It is straighforward to verify that  the resulting solutions are completely smooth and have the expected vanishing of $e^{2f}$ at the origin, implying that the spacetime smoothly pinches off. Furthermore, $e^{2f}/e^{2u}$ is seen to asymptote to a constant, which we denote by $e^{2 f_k}$.
\subsection{UV asymptotic expansions}
As in the holographic Janus solutions in Lorentzian signature , the BPS equations may also be used to obtain the UV asymptotic behavior of the solutions. To do so, we begin by defining an asymptotic coordinate $z = e^{- u}$, where the asymptotic $S^5$  boundary is reached by taking $u\to \infty$. Consequently, an asymptotic expansion is an expansion around $z=0$. The coefficients in the UV expansions of the non-zero fields may now be solved for order-by-order using the BPS equations. One finds explicitly that all coefficients are determined in terms of only three independent parameters $\alpha$, $\beta$, and $f_k$, in accord with the fact that there are three independent first-order differential equations. The first few terms in the expansions are
\begin{eqnarray}
f(z) = -\log z +f_k -  \left({1\over 4}e^{-2f_k} + {1\over 16}\alpha^2\right)  \,z^2 + O(z^4) \nonumber\\
\sigma(z) = {3\over 8} \alpha^2 \,z^2 +  {1\over 4} e^{f_k} \alpha \beta \,z^3 + O(z^4) 
\nonumber\\
\phi^0(z) = \alpha \, z - \left({5\over 4}\alpha \,e^{-2f_k} + {23\over 48} \alpha^3\right) \,z^3 + O(z^4)
\nonumber\\
\phi^3(z) = e^{-f_k} \alpha  z^2 + \beta \, z^3 + O(z^4) 
\end{eqnarray}
We have obtained the expansions up to $O(z^8)$, but we display only the first few terms here.
\section{Holographic sphere free energy}
\setcounter{equation}{0}
The goal of this section is to obtain the holographic free energy, i.e. the renormalized on-shell action. We begin by writing the full action, 
\begin{eqnarray}
\vphantom{.} \hspace{1.7 in} S = S_{\mathrm{6D}} + S_{\mathrm{GH}}
\nonumber\\\nonumber\\
\vphantom{.} S_{\mathrm{6D}} = \int du~d^5 x ~\sqrt{G}\, {\cal L} \hspace{1 in}S_{\mathrm{GH}}=- {1 \over 2} \int d^5 x \sqrt{\gamma}\, {\cal K}
\end{eqnarray}
where $S_{\mathrm{6D}}$ is the six-dimensional Euclidean action given in and $S_{\mathrm{GH}}$ is the Gibbons-Hawking term. The $\gamma$ appearing in $S_{\mathrm{GH}}$ is the determinant of the induced metric on the boundary 
\end{document}
