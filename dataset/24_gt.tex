\documentclass[12pt]{article}
\setlength{\topmargin}{-.3in}
\setlength{\oddsidemargin}{0in}
\setlength{\textheight}{8.2in}
\setlength{\textwidth}{6.5in}
\setlength{\footnotesep}{\baselinestretch\baselineskip}
\newlength{\abstractwidth}
\setlength{\abstractwidth}{\textwidth}
\addtolength{\abstractwidth}{-6pc}
\usepackage{amsmath}
\usepackage{amsfonts}
\usepackage{amssymb}
\usepackage{latexsym}
\usepackage{epsf}
\usepackage{color}
\usepackage{graphicx}
\usepackage{tikz}
\usepackage{dsfont}
\usepackage{subfigure}
\usepackage{hyperref}
\pagestyle{plain}
\begin{document}

 $e_i \pm e_j$ for all $i \neq j$. The free energy in the specific case of a vector multiplet in the adjoint, a single antisymmetric hypermultiplet, and $N_f$ fundamental hypermultiplets then is
\begin{eqnarray}
F(\lambda_i) = \sum_{i \neq j} \left[F_V(\lambda_i - \lambda_j) + F_V(\lambda_i + \lambda_j) + F_H(\lambda_i - \lambda_j) + F_H(\lambda_i + \lambda_j) \right] 
\nonumber\\
\vphantom{.}\hspace{0.3in}+ \sum_i \left[F_V(2 \lambda_i) + F_V(-2 \lambda_i) + N_f F_H(\lambda_i) + N_f F_H(-\lambda_i) \right]
\end{eqnarray}
The next step is to look for extrema of this function in the specific case of $\lambda_i \geq 0$ for all $i$. Extrema in the case of non-positive $\lambda_i$ can be obtained from these through action of the Weyl group.
To calculate the extrema, one first assumes that as $N \rightarrow \infty$, the vevs scale as $\lambda_i = N^\alpha x_i$ for $\alpha>0$ and $x_i$ of order $O(N^0)$. One then introduces a density function
\begin{eqnarray}
\rho(x) = {1 \over N} \sum_{i=1}^N \delta(x-x_i) 
\end{eqnarray}
which in the continuum limit should approach an $L^1$ function normalized as 
\begin{eqnarray}
\int dx\, \rho(x) = 1
\end{eqnarray}
In terms of this density function, one finds that 
\begin{eqnarray}
F \approx -{9 \pi \over 8} N^{2 + \alpha} \int dx dy \,\rho(x) \rho(y) \left(|x-y| + |x+y| \right) + {\pi (8-N_f) \over 3} N^{1 + 3 \alpha} \int dx\, \rho(x)\, |x|^3
\end{eqnarray}
where the large argument expansions  have been used, and terms subleading in $N$ have been dropped. This only has non-trivial saddle points when both terms scale the same with $N$, which demands that $\alpha=1/2$ and gives the famous result that $F\propto N^{5/2}$. Extremizing the free energy over normalized density functions then gives 
\begin{eqnarray}
F \approx - {9 \sqrt{2} \pi N^{5/2} \over 5 \sqrt{8-N_f}}
\end{eqnarray}
This value of the free energy is to be identified with the renormalized on-shell action of the supersymmetric AdS$_6$ solution. This identification yields the following relation between the six-dimensional Newton's constant $G_6$ and the parameters $N$ and $N_f$ of the dual SCFT,
\begin{eqnarray}
G_6= \frac{5\pi\sqrt{8-N_f}}{27\sqrt{2}} ~N^{-5/2}
\end{eqnarray}
\end{document}
