\documentclass[a4paper,12pt]{article}
\usepackage{latexsym}
\usepackage{amsmath}
\usepackage{amssymb}
\usepackage{graphicx}
\usepackage{wrapfig}
\pagestyle{plain}
\usepackage{fancybox}
\usepackage{bm}

\begin{document}

P1 Affine invariance: $d(\mathrm{x};\mathrm{F})=\mathrm{d}(\mathrm{A}\mathrm{x}+\mathrm{b};\mathrm{F}_{\mathrm{A},\mathrm{b}})$ ;

P2 Maximality at center: if $F$ is “symmetric” around µ then $d(\mathrm{x};\mathrm{F}) \leq$

d(µ; F) for all $\mathrm{x}$; for a more detailed discussion on symmetry see ?.

P3 Monotonicity: if $()$ holds, then

$ d(\mathrm{x};\mathrm{F})\leq$ d(µ $+$a(x--µ); F) $\ovalbox{\tt\small REJECT}\in[0$, 1$]$ ;

P4 Approaching zero: $\Vert \mathrm{x}\Vert\rightarrow\infty\Rightarrow \mathrm{d}(\mathrm{x};\mathrm{F})\rightarrow 0.$

1 Gervini-Yohai depth

Here we want to show that the Gervini-Yohai depth, defined as $d_{GY}(\mathrm{t},\ \mathrm{F},\ \mathrm{G})=$

$1-\mathrm{G}$ ($\Delta$ ($\mathrm{t}$, µ(F), $\Sigma(\mathrm{F})$)), is a proper statistical depth function, i.e., it satis-

fies the four properties introduced above.

1. Affine invariance: it follows directly from the affine invariance property

of the Mahalanobis distance;

2. Maximality at center: if $F$ is elliptically symmetric around µ(F),
\begin{center}
$d_{GY}$ (µ($\mathrm{F}$), $\mathrm{F}, \mathrm{G}$) $=1-\mathrm{G}$ ($\Delta$(µ($\mathrm{F}$), µ($\mathrm{F}$), $\Sigma(\mathrm{F}))$) $=1-\mathrm{G}(0)$ .
\end{center}
For any $\mathrm{t}\neq$ µ(F) we have
\begin{center}
$\Delta$ ($\mathrm{t}$, µ($\mathrm{F}$), $\Sigma(\mathrm{F})$) $>0$

$G$ ($\Delta$ ($\mathrm{t}$, µ($\mathrm{F}$), $\Sigma(\mathrm{F}))$) $\geq \mathrm{G}(0)$

$1-G$ ($\Delta$ ($\mathrm{t}$, µ($\mathrm{F}$), $\Sigma(\mathrm{F}))$) $\leq 1-\mathrm{G}(0)$

$d_{GY}(\mathrm{t},\ \mathrm{F},\ \mathrm{G})\leq \mathrm{d}_{\mathrm{G}\mathrm{Y}}$ (µ($\mathrm{F}$), $\mathrm{F}, \mathrm{G}$),
\end{center}
when $G$ is strictly monotone then strict inequality holds, and µ(F) is

the unique maximizer of the Gervini-Yohai depth.

3. Monotonicity:

$\Delta$(µ(F) $+$a(t --µ(F)), µ(F), $\Sigma(\mathrm{F})$) $=$ (a(t --µ(F)))T $\Sigma(\mathrm{F})^{-1}$ (a(t --µ(F)))

$=\ovalbox{\tt\small REJECT} 2$ ($\mathrm{t}$--µ(F))T $\Sigma(\mathrm{F})^{-1}$($\mathrm{t}$--µ(F))

$=$ a2$\Delta$(t, µ(F), $\Sigma(\mathrm{F})$)

$\leq\Delta$($\mathrm{t}$, µ(F), $\Sigma(\mathrm{F})$)

Then $d_{GY}$ (µ(F) $+$a(t --µ(F)), $\mathrm{F}, \mathrm{G}$) $\geq \mathrm{d}_{\mathrm{G}\mathrm{Y}}(\mathrm{t},\ \mathrm{F},\ \mathrm{G})$ .

4. Approaching zero: if $\Vert \mathrm{t} \Vert\rightarrow \infty$ we have that $\Delta$ ($\mathrm{t}$, µ(F), $\Sigma(\mathrm{F})$) $\rightarrow \infty$

and consequently $G$ ($\Delta$ ($\mathrm{t}$, µ(F), $\Sigma(\mathrm{F}))$) $\rightarrow 1$. Then
\begin{center}
$d_{GY}(\mathrm{t},\ \mathrm{F},\ \mathrm{G})=1-\mathrm{G}$($\Delta$ ($\mathrm{t}$, µ($\mathrm{F}$), $\Sigma(\mathrm{F}))$) $\rightarrow 0$
\end{center}
1
\end{document}
